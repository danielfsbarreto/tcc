\chapter*{Resumo}

As metodologias ágeis de desenvolvimento visam construir software de forma mais rápida e eficiente, criando um produto mínimo que agrega valor ao cliente e, partir daí, incrementando-o com novas funcionalidades. Esta é uma grande quebra de paradigma se as compararmos com as metodologias orientadas a planejamento.

Mudar drasticamente nunca é fácil. Diante deste cenário, surgiu uma linha de pesquisa voltada para estudo deste processo de transição de metodologias "plan-driven" para Ágil. Segundo estas pesquisas, os principais benefícios com a adoção de métodos ágeis são: satisfação do cliente, entregas mais frequentes, manutenção da moral do time elevada, melhoria na qualidade do produto final, etc. Contudo, conquistar todos estes benefícios requer muita disciplina e dedicação.

Analisando os resultados obtidos com a pesquisa, percebeu-se a existência de um conjunto de lições aprendidas mais comuns. Dentre elas, podemos citar: ``customização e adaptabilidade", ``experiência, treinamento e aprendizado", ``engajamento, comprometimento, disciplina e trabalho em equipe" e ``aspectos técnicos e tecnológicos".

\textbf{Palavras-chave:} Métodos ágeis, adoção ágil, lições aprendidas

