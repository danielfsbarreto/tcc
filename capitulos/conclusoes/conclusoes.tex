\chapter{Conclusão da pesquisa}

Nesse capítulo consta a conclusão da pesquisa, bem como possíveis trabalhos futuros.

\section{Conclusão}

A cada dia, mais organizações de desenvolvimento de software estão modificando seus processos internos e adotando metodologias ágeis. Esta mudança ocorre devido a uma contínua busca por melhorias na produtividade e qualidade dos softwares construídos. Entretanto, modificar processos não é uma tarefa fácil, exige esforço e dedicação das diversas partes envolvidas, pois geralmente é preciso enfrentar muita resistência durante a transição.

Apoio gerencial e dos clientes, problemas com novas tecnologias, comprometimento e disciplina, incompatibilidades entre princípios (Ágil x Organização) e falta de experiência foram alguns dos desafios mais encontrados nos relatos pesquisados. A maioria dessas dificuldades pode ser resolvida através do compartilhamento de conhecimento. Como é possível evitar cometer erros aprendendo através de relatos de experiências compartilhados por outras organizações, a proposta dessa pesquisa foi criar uma fonte com informações úteis relacionadas a processos de adoção ágil em empresas de desenvolvimento de software.

As lições aprendidas encontradas foram agrupadas em 14 categorias validadas, através de um questionário, por especialistas atuantes no mercado de TI brasileiro. Das 14 categorias, 12 foram consideradas de máxima prioridade por, no mínimo, 80\% dos profissionais consultados. Dentre as mais importantes, podemos citar: ``Engajamento, comprometimento, disciplina e trabalho em equipe", ``Cultura organizacional" e ``Experiência, treinamento e aprendizado".

\section{Trabalhos futuros}

A partir desse estudo, uma série de outros trabalhos podem surgir. O mais imediato dentre eles seria um conjunto de artigos a serem publicados e apresentados nas principais conferências ágeis do Brasil em 2014. Outra contribuição, porém de um prazo mais longo, é a criação de uma plataforma colaborativa open-source de relatos de experiência e lições aprendidas com adoção ágil.
