\chapter{Análise qualitativa de lições aprendidas}

Após a seleção do conjunto de trabalhos primários desta pesquisa (16 artigos científicos e 14 relatos de experiência), as lições aprendidas reportadas por eles foram devidamente mapeadas e categorizadas. Neste capítulo encontra-se o resultado deste mapeamento.

\section{Mapeamento e categorização}
Para obter-se uma maior visibilidade do que foi encontrado, as diversas lições aprendidas reportadas pela base da literatura científica e relatos de experiência foram multidimensionalmente mapeadas.

Foi considerada a existência real de uma categoria quando pelo menos 6 trabalhos (20\% do material analisado) referenciavam alguma lição aprendida que se encaixasse naquela categoria (Tabela \ref{tab:mapeamentoCategorias}).

\begin{table}[H]
	\centering
	\begin{tabularx}{\linewidth}{ | p{6cm} | X | X | }
		\hline 
		\textbf{Categorias} & \textbf{Artigos científicos} & \textbf{Relatos de experiência} \\ 
		\hline 
		Experiência, treinamento e aprendizado & \cite{Hajjdiab2011}, \cite{Block2011}, \cite{Adobe2012}, \cite{Cisco2011}, \cite{Lapham2012}, \cite{Eunha2012}, \cite{Claudia2013}, \cite{Asnawi2012}, \cite{Fitzgerald2013} & \cite{Stefano2013}, \cite{Rodrigues2013}, \cite{Bastos2013}, \cite{Maciel2013}, \cite{Karaj2013}, \cite{Piegas2012}, \cite{Vieira2013} \\ 
		\hline 
		Planejamento e gerenciamento de backlog & \cite{Hajjdiab2011}, \cite{Fitzgerald2013}, \cite{Block2011}, \cite{Adobe2012}, \cite{Bustard2013}, \cite{Korhonen2010}, \cite{Claudia2013} & \cite{Piegas2012}, \cite{Hui2013}, \cite{Parzinello2012} \\ 
		\hline 
		Apoio gerencial e dos clientes & \cite{Hajjdiab2011}, \cite{Cisco2011}, \cite{Claudia2013}, \cite{Arikpo2011} & \cite{Parzinello2012}, \cite{Stefano2013}, \cite{Bastos2013}, \cite{Maciel2013}, \cite{Srinath2012}, \cite{Piegas2012} \\ 
		\hline 
		Customização e adaptabilidade & \cite{Hajjdiab2011}, \cite{Block2011}, \cite{Asnawi2012}, \cite{Fitzgerald2013}, \cite{Bustard2013}, \cite{Microsoft2013}, \cite{Lapham2012}, \cite{Claudia2013}, \cite{Nokia2013} & \cite{Piegas2012}, \cite{Hui2013}, \cite{Rodrigues2013}, \cite{Bastos2013}, \cite{Maciel2013}, \cite{Ahmed2008}, \cite{Sahota2012}, \cite{Vieira2013} \\ 
		\hline 
		Confiança do time & \cite{Block2011}, \cite{Asnawi2012}, \cite{Claudia2013}, \cite{Nokia2013} & \cite{Parzinello2012}, \cite{Ahmed2008}, \cite{Piegas2012}, \cite{Bastos2013} \\ 
		\hline 
		Engajamento, comprometimento, disciplina e trabalho em equipe & \cite{Block2011}, \cite{Asnawi2012}, \cite{Lapham2012}, \cite{Microsoft2013}, \cite{Claudia2013}, \cite{Nokia2013}, \cite{Adobe2012}, \cite{Fitzgerald2013} & \cite{Piegas2012}, \cite{Parzinello2012}, \cite{Stefano2013}, \cite{Rodrigues2013}, \cite{Maciel2013}, \cite{Queiroz2013}, \cite{Bastos2013}, \cite{Ahmed2008} \\ 
		\hline 
		Aspectos técnicos e tecnológicos & \cite{Block2011}, \cite{Microsoft2013}, \cite{Korhonen2010}, \cite{Cisco2011}, \cite{Lapham2012}, \cite{Eunha2012}, \cite{Fitzgerald2013}, \cite{Arikpo2011}, \cite{Bustard2013}, \cite{Radha2012}, \cite{Nokia2013} & \cite{Piegas2012}, \cite{Queiroz2013}, \cite{Stefano2013}, \cite{Karaj2013} \\ 
		\hline 
		Compartilhamento de conhecimento & \cite{Asnawi2012}, \cite{Cisco2011}, \cite{Lapham2012}, \cite{Radha2012}, \cite{Eunha2012}, \cite{Ericsson2013} & \cite{Valerio2013}, \cite{Vieira2013}, \cite{Queiroz2013}, \cite{Bastos2013}, \cite{Maciel2013} \\ 
		\hline 
		Velocidade de entrega e produtividade & \cite{Adobe2012}, \cite{Fitzgerald2013}, \cite{Microsoft2013}, \cite{Cisco2011}, \cite{Korhonen2010}, \cite{Eunha2012}, \cite{Claudia2013} & \cite{Stefano2013}, \cite{Queiroz2013}, \cite{Maciel2013}, \cite{Hui2013}, \cite{Ahmed2008}, \cite{Piegas2012} \\ 
		\hline 
		Qualidade do produto final & \cite{Adobe2012}, \cite{Fitzgerald2013}, \cite{Bustard2013}, \cite{Lapham2012}, \cite{Eunha2012}, \cite{Claudia2013}, \cite{Korhonen2010} & \cite{Parzinello2012}, \cite{Maciel2013}, \cite{Ahmed2008} \\ 
		\hline 
		Tamanho da organização & \cite{Bustard2013}, \cite{Microsoft2013}, \cite{Claudia2013}, \cite{Korhonen2010}, \cite{Ericsson2013} & \cite{Maciel2013} \\ 
		\hline 
		Quebra de paradigma & \cite{Hajjdiab2011}, \cite{Block2011}, \cite{Korhonen2010}, \cite{Lapham2012}, \cite{Arikpo2011} & \cite{Stefano2013}, \cite{Bastos2013}, \cite{Maciel2013}, \cite{Parzinello2012}, \cite{Hui2013}, \cite{Ahmed2008}, \cite{Sahota2012} \\ 
		\hline 
		Comunicação remota & \cite{Adobe2012}, \cite{Microsoft2013}, \cite{Korhonen2010}, \cite{Radha2012} & \cite{Rodrigues2013}, \cite{Vieira2013}, \cite{Bastos2013}, \cite{Maciel2013} \\ 
		\hline 
		Cultura organizacional & \cite{Bustard2013}, \cite{Microsoft2013}, \cite{Claudia2013}, \cite{Nokia2013}, \cite{Eunha2012}, \cite{Fitzgerald2013} & \cite{Rodrigues2013}, \cite{Bastos2013}, \cite{Srinath2012}, \cite{Maciel2013}, \cite{Sahota2012} \\ 
		\hline 
	\end{tabularx}
	\caption{Mapeamento de categorias de lições aprendidas e suas respectivas referências}
	\label{tab:mapeamentoCategorias}
\end{table}

\section{Lições aprendidas}

\subsection{Experiência, treinamento e aprendizado}

\begin{table}[H]
	\centering
	\begin{tabularx}{\linewidth}{ | X | p{5cm} | } \hline \textbf{Lições aprendidas} & \textbf{Referências} \\ \hline
		É muito difícil aventurar-se em ágil sem um Agile Coach & \cite{Hajjdiab2011}, \cite{Block2011}, \cite{Adobe2012}, \cite{Cisco2011}, \cite{Lapham2012}, \cite{Eunha2012}, \cite{Claudia2013}, \cite{Stefano2013}, \cite{Rodrigues2013}, \cite{Bastos2013}, \cite{Maciel2013}, \cite{Karaj2013} \\ \hline
		Vivenciar um projeto-piloto é uma prática muito eficiente para se adquirir experiência & \cite{Hajjdiab2011}, \cite{Cisco2011}, \cite{Rodrigues2013}, \cite{Maciel2013} \\ \hline
		É preciso prática, não apenas estudo & \cite{Hajjdiab2011}, \cite{Asnawi2012}, \cite{Claudia2013}, \cite{Piegas2012}, \cite{Vieira2013} \\ \hline
		Trabalhar com equipes não-agéis pode atrapalhar o andamento do projeto & \cite{Adobe2012}, \cite{Maciel2013} \\ \hline
	\end{tabularx}
\end{table}

Por muitos anos empresas de desenvolvimento de software adotaram o modelo cascata como forma de gerenciamento de projetos. O Manifesto Ágil \cite{agileManifesto}, ocorrido em 2001, modificou completamente a maneira de lidar com esse tipo de atividade.

Mudar bruscamente nunca é simples. Para que esta transição não seja tão dolorosa, uma tática muito popular entre as empresas é a contratação de um profissional experiente para servir como guia e treinar toda a equipe. Segundo \cite{Hajjdiab2011}, este papel é essencial durante o processo de adoção ágil em qualquer organização. Outra maneira de se adquirir experiência de forma rápida e eficaz é através de um projeto-piloto. \cite{Cisco2011} utilizou esta abordagem para entender melhor como a empresa funcionava e atacar nos pontos mais críticos.

Alguns pontos negativos também foram levantados nos trabalhos analisados. Para \cite{Piegas2012}, é preciso aprender com os próprios erros, não apenas realizar treinamentos. E, de acordo com \cite{Adobe2012} e \cite{Maciel2013}, trabalhar com stakeholders que não são ágeis pode afetar negativamente o desempenho do time ágil (que trabalha em um ambiente menos burocrático e com um ciclo de feedback bem mais curto).

\subsection{Planejamento e gerenciamento de backlog}

\begin{table}[H]
	\centering
	\begin{tabularx}{\linewidth}{ | X | p{5cm} | } \hline \textbf{Lições aprendidas} & \textbf{Referências} \\ \hline
		Planejar apenas quando necessário & \cite{Hajjdiab2011}, \cite{Fitzgerald2013}, \cite{Piegas2012}, \cite{Hui2013}, \cite{Parzinello2012} \\ \hline
		O processo de adoção em si precisa ser bem planejado & \cite{Hajjdiab2011} \\ \hline
		Priorizar o backlog é uma tarefa complicada para times inexperientes & \cite{Block2011} \\ \hline
		É difícil lidar com o aumento de escopo & \cite{Block2011} \\ \hline
		É preciso aprender a quebrar o backlog da forma correta & \cite{Adobe2012}, \cite{Hui2013}, \cite{Parzinello2012} \\ \hline
		A coleta e gerência de requisitos ocorre de forma bem diferente do habitual & \cite{Bustard2013}, \cite{Korhonen2010}, \cite{Claudia2013}, \cite{Piegas2012}, \cite{Hui2013} \\ \hline
	\end{tabularx}
\end{table}

Desenvolver de forma iterativa e incremental gera um grande impacto na forma de gerenciamento de projetos ágeis. Não é necessário planejar todas as etapas do processo, é difícil prever situações e cenários muito distantes. Planejar apenas o necessário, quando necessário, é um grande desafio para muitas empresas. O paper \cite{Fitzgerald2013} relata que teve que lidar com muitos problemas relacionados à granularidade no planejamento que Ágil propõe.

Todavia, não podemos confundir o processo planejamento utilizado em projetos ágeis e o planejamento para se adotar ágil. Dado que esta é uma grande mudança de paradigma, é preciso ter cautela. \cite{Hajjdiab2011} aconselha que este processo deve ser bem planejado e que aconteça de forma gradual.

Outro ponto considerado desafiador por muitas organizações é o gerenciamento de backlog. O artigo \cite{Block2011} relata que passou por dificuldades ao tentar priorizá-lo e impedir que ele crescesse além do comportado pela equipe. Ainda com relação ao backlog, \cite{Adobe2012} lembra que um dos princípios primários do desenvolvimento ágil de software é o foco na entrega de pequenos incrementos de valor. Isto pode até parecer simples à primeira impressão, porém dividir todo um conjunto de requisitos de um projeto em pequenas fatias independentes entre si e que agregam valor ao cliente não é algo trivial.

\subsection{Apoio gerencial e dos clientes}

\begin{table}[H]
	\centering
	\begin{tabularx}{\linewidth}{ | X | p{5cm} | } \hline \textbf{Lições aprendidas} & \textbf{Referências} \\ \hline
		Pressão por parte do alto escalão da empresa afeta negativamente no andamento do processo de adoção ágil & \cite{Hajjdiab2011}, \cite{Cisco2011}, \cite{Claudia2013}, \cite{Parzinello2012}, \cite{Stefano2013}, \cite{Bastos2013}, \cite{Maciel2013}, \cite{Srinath2012} \\ \hline
		O apoio do cliente é de suma importância para o sucesso do projeto & \cite{Arikpo2011}, \cite{Claudia2013}, \cite{Parzinello2012}, \cite{Stefano2013}, \cite{Maciel2013} \\ \hline
		É muito difícil contornar uma situação quando o cliente pressiona o tima para que se adote práticas waterfall & \cite{Claudia2013}, \cite{Piegas2012}, \cite{Srinath2012} \\ \hline
		É preciso ter liberdade ao se adotar ágil & \cite{Piegas2012}, \cite{Stefano2013}, \cite{Maciel2013} \\ \hline
	\end{tabularx}
\end{table}

Houve uma unanimidade quanto a este ponto. É preferível que todos os envolvidos em projetos de desenvolvimento de software que utilizam alguma metodologia ágil estejam alinhados com o processo. O apoio da gerência e dos clientes é fundamental para o bom desempenho da equipe.

Mais precisamente, em muitos casos, apenas o apoio não é o suficiente. É preciso ter liberdade para se adotar ágil de forma bem sucedida. Muitos materiais \cite{Piegas2012,Stefano2013,Maciel2013} relataram ter tido problemas com esse nível de alinhamento com os princípios ágeis de liberdade e auto-organização.

\subsection{Customização e adaptabilidade}

\begin{table}[H]
	\centering
	\begin{tabularx}{\linewidth}{ | X | p{5cm} | } \hline \textbf{Lições aprendidas} & \textbf{Referências} \\ \hline
		Ágil é customizável, não existem regras no seu processo de adoção & \cite{Hajjdiab2011}, \cite{Piegas2012}, \cite{Hui2013} \\ \hline
		Dar suporte para adaptabilidade é um problema não-trivial & \cite{Block2011}, \cite{Asnawi2012}, \cite{Fitzgerald2013}, \cite{Bustard2013}, \cite{Microsoft2013}, \cite{Lapham2012}, \cite{Claudia2013}, \cite{Nokia2013}, \cite{Rodrigues2013}, \cite{Bastos2013}, \cite{Maciel2013}, \cite{Hui2013}, \cite{Ahmed2008}, \cite{Sahota2012} \\ \hline
		Dificuldades na implantação de mudanças necessárias & \cite{Vieira2013}, \cite{Bastos2013}, \cite{Maciel2013}, \cite{Hui2013}, \cite{Sahota2012} \\ \hline
	\end{tabularx}
\end{table}

Um dos principais pilares do desenvolvimento ágil é encarar mudanças como bem-vindas. Todavia, atingir um nível de maturidade de tal forma que isto ocorra naturalmente é algo para se orgulhar. Muitos trabalhos relataram problemas para dar suporte a esta adaptabilidade. Em muitas situações, times encontram dificuldades para implantar as mudanças necessárias para obter-se um melhor resultado final em seus projetos. Um exemplo claro deste fator é o demonstrado por \cite{Fitzgerald2013}. Segundo seus autores, ambientes regulados e métodos ágeis são frequentemente vistos como fundamentalmente incompatíveis, o que causa uma série de problemas deste gênero.

Outro ponto relevante levantado por alguns trabalhos é o quanto ágil pode ser customizável. Não exite um conjunto pré-definido de regras que devem ser seguidas à risca por aqueles que querem ser ágeis. Ágil é flexível. Segundo \cite{Hajjdiab2011}, um dos principais benefícios do desenvolvimento ágil é a sua capacidade de ser personalizado baseado na cultura e no ambiente da organização que o está adotando.

\subsection{Confiança do time}

\begin{table}[H]
	\centering
	\begin{tabularx}{\linewidth}{ | X | p{5cm} | } \hline \textbf{Lições aprendidas} & \textbf{Referências} \\ \hline
		Entregar mais e entregar valor  ajuda a manter o time confiante & \cite{Block2011}, \cite{Asnawi2012}, \cite{Parzinello2012} \\ \hline
		Utilizar métodos ágeis ajuda a manter elevada a moral do time & \cite{Asnawi2012}, \cite{Claudia2013}, \cite{Nokia2013}, \cite{Ahmed2008} \\ \hline
		A manutenção da confiança das partes envolvidas mostrou-se comprometida com o uso de metodologias ágeis & \cite{Nokia2013}, \cite{Piegas2012}, \cite{Bastos2013} \\ \hline
	\end{tabularx}
\end{table}

\subsection{Engajamento, comprometimento, disciplina e trabalho em equipe}

\begin{table}[H]
	\centering
	\begin{tabularx}{\linewidth}{ | X | p{5cm} | } \hline \textbf{Lições aprendidas} & \textbf{Referências} \\ \hline
		É difícil encontrar pessoas que trabalham bem em equipe & \cite{Block2011} \\ \hline
		É muito importante ter o PO próximo ou, se possível, como membro ativo do time, envolvido em todas as etapas do processo & \cite{Block2011}, \cite{Asnawi2012}, \cite{Lapham2012}, \cite{Microsoft2013}, \cite{Claudia2013}, \cite{Piegas2012}, \cite{Parzinello2012}, \cite{Stefano2013}, \cite{Rodrigues2013}, \cite{Maciel2013} \\ \hline
		Falta de ownership no projeto & \cite{Block2011}, \cite{Nokia2013}, \cite{Queiroz2013} \\ \hline
		O envolvimento dos desenvolvedores dentro do processo é extremamente importante & \cite{Asnawi2012}, \cite{Adobe2012}, \cite{Fitzgerald2013}, \cite{Lapham2012}, \cite{Microsoft2013}, \cite{Claudia2013}, \cite{Stefano2013}, \cite{Bastos2013}, \cite{Maciel2013}, \cite{Ahmed2008} \\ \hline
		Para projetos ágeis serem bem sucedidos, é preciso disciplina & \cite{Parzinello2012} \\ \hline
	\end{tabularx}
\end{table}

\nomenclature{PO}{Product Owner}%

\subsection{Aspectos técnicos e tecnológicos}

\begin{table}[H]
	\centering
	\begin{tabularx}{\linewidth}{ | X | p{5cm} | } \hline \textbf{Lições aprendidas} & \textbf{Referências} \\ \hline
		Utilizar ágil em um projeto com código legado mostrou-se desafiador & \cite{Block2011} \\ \hline
		Integração contínua, automação de build e testes automatizados promovem ganhos muito vantajosos & \cite{Block2011}, \cite{Microsoft2013}, \cite{Korhonen2010}, \cite{Cisco2011}, \cite{Lapham2012}, \cite{Eunha2012} \\ \hline
		É necessário o suporte de boas ferramentas & \cite{Fitzgerald2013}, \cite{Microsoft2013}, \cite{Arikpo2011} \\ \hline
		A atividade de design arquitetural mostrou-se desafiadora & \cite{Bustard2013}, \cite{Radha2012}, \cite{Piegas2012} \\ \hline
		Ágil promove uma melhor manutenibilidade do código & \cite{Bustard2013} \\ \hline
		Infra-estruturas virtualizadas proporcionam a flexibilidade necessária para muitos projetos ágeis & \cite{Radha2012} \\ \hline
		Ágil promove uma pior manutenibilidade do código & \cite{Nokia2013}, \cite{Queiroz2013} \\ \hline
		Em certos casos há resistência por parte do cliente para utilizar o produto construído & \cite{Stefano2013} \\ \hline
		Pensamento ágil precisa vir com capacidade técnica & \cite{Karaj2013} \\ \hline
	\end{tabularx}
\end{table}

\subsection{Compartilhamento de conhecimento}

\begin{table}[H]
	\centering
	\begin{tabularx}{\linewidth}{ | X | p{5cm} | } \hline \textbf{Lições aprendidas} & \textbf{Referências} \\ \hline
		Compartilhar conhecimento e experiências é crucial para o sucesso de projetos ágeis & \cite{Asnawi2012}, \cite{Cisco2011}, \cite{Lapham2012}, \cite{Radha2012}, \cite{Eunha2012}, \cite{Valerio2013}, \cite{Vieira2013}, \cite{Queiroz2013}, \cite{Bastos2013}, \cite{Maciel2013} \\ \hline
		O ambiente com uma cultura ágil favorece o compartilhamento de conhecimento & \cite{Ericsson2013} \\ \hline
	\end{tabularx}
\end{table}

\subsection{Velocidade de entrega e produtividade}

\begin{table}[H]
	\centering
	\begin{tabularx}{\linewidth}{ | X | p{5cm} | } \hline \textbf{Lições aprendidas} & \textbf{Referências} \\ \hline
		Há um aumento na velocidade/frequência de entrega com ágil & \cite{Adobe2012}, \cite{Fitzgerald2013}, \cite{Microsoft2013}, \cite{Cisco2011}, \cite{Korhonen2010}, \cite{Eunha2012}, \cite{Claudia2013}, \cite{Stefano2013}, \cite{Queiroz2013}, \cite{Maciel2013}, \cite{Hui2013}, \cite{Ahmed2008} \\ \hline
		Defeitos são corrigidos mais rapidamente & \cite{Microsoft2013}, \cite{Korhonen2010} \\ \hline
		A velocidade do time é variável & \cite{Piegas2012} \\ \hline
	\end{tabularx}
\end{table}

\subsection{Qualidade do produto final}

\begin{table}[H]
	\centering
	\begin{tabularx}{\linewidth}{ | X | p{5cm} | } \hline \textbf{Lições aprendidas} & \textbf{Referências} \\ \hline
		A melhoria na qualidade e valor agregado do produto é notória ao se desenvolvê-lo utilizando ágil & \cite{Adobe2012}, \cite{Fitzgerald2013}, \cite{Bustard2013}, \cite{Lapham2012}, \cite{Eunha2012}, \cite{Claudia2013}, \cite{Parzinello2012}, \cite{Maciel2013}, \cite{Ahmed2008} \\ \hline
		A diferente maneira de testar a aplicação reduz riscos & \cite{Korhonen2010}, \cite{Lapham2012}, \cite{Eunha2012}, \cite{Parzinello2012}, \cite{Ahmed2008} \\ \hline
	\end{tabularx}
\end{table}

\subsection{Tamanho da organização}

\begin{table}[H]
	\centering
	\begin{tabularx}{\linewidth}{ | X | p{5cm} | } \hline \textbf{Lições aprendidas} & \textbf{Referências} \\ \hline
		Pequenas empresas conseguem obter benefícios mais facilmente ao adotar ágil & \cite{Bustard2013} \\ \hline
		Escalar ágil para empresas ou projetos maiores é complicado, porém não é impossível & \cite{Microsoft2013}, \cite{Claudia2013}, \cite{Korhonen2010}, \cite{Maciel2013} \\ \hline
		É possível implementar ágil parcialmente & \cite{Ericsson2013} \\ \hline
	\end{tabularx}
\end{table}

\subsection{Quebra de paradigma}

\begin{table}[H]
	\centering
	\begin{tabularx}{\linewidth}{ | X | p{5cm} | } \hline \textbf{Lições aprendidas} & \textbf{Referências} \\ \hline
		É difícil lidar com a quebra de paradigma para a implementação de ágil & \cite{Hajjdiab2011}, \cite{Block2011}, \cite{Korhonen2010}, \cite{Lapham2012}, \cite{Arikpo2011}, \cite{Stefano2013}, \cite{Bastos2013}, \cite{Maciel2013} \\ \hline
		A maneira de gerenciar/implementar os testes de softwares construídos em projeto ágeis é desafiadora & \cite{Korhonen2010} \\ \hline
		Colocar princípios acima de práticas & \cite{Maciel2013}, \cite{Parzinello2012}, \cite{Hui2013}, \cite{Ahmed2008}, \cite{Sahota2012} \\ \hline
	\end{tabularx}
\end{table}

\subsection{Comunicação remota}

\begin{table}[H]
	\centering
	\begin{tabularx}{\linewidth}{ | X | p{5cm} | } \hline \textbf{Lições aprendidas} & \textbf{Referências} \\ \hline
		Coordernar atividades com times distribuídos é algo bastante desafiador & \cite{Adobe2012}, \cite{Microsoft2013}, \cite{Korhonen2010}, \cite{Radha2012}, \cite{Rodrigues2013}, \cite{Vieira2013}, \cite{Bastos2013}, \cite{Maciel2013} \\ \hline
		Sincronizar board virtual e real não é uma tarefa fácil & \cite{Vieira2013} \\ \hline
		Dificuldade ao trabalhar com equipes externas de testes & \cite{Bastos2013} \\ \hline
	\end{tabularx}
\end{table}

\subsection{Cultura organizacional}

\begin{table}[H]
	\centering
	\begin{tabularx}{\linewidth}{ | X | p{5cm} | } \hline \textbf{Lições aprendidas} & \textbf{Referências} \\ \hline
		Métodos ágeis e ambiente altamente controlados são, em sua essência, culturalmente incompatíveis & \cite{Fitzgerald2013} \\ \hline
		Empresas que discordam de princípios ágeis tendem a falhar caso tentem implantar tais métodos & \cite{Bustard2013}, \cite{Microsoft2013}, \cite{Claudia2013}, \cite{Nokia2013}, \cite{Sahota2012}, \cite{Maciel2013} \\ \hline
		Tentar influenciar/remodelar a cultura da organização é uma tarefa desafiadora & \cite{Eunha2012}, \cite{Rodrigues2013}, \cite{Bastos2013}, \cite{Sahota2012}, \cite{Srinath2012}, \cite{Maciel2013} \\ \hline
	\end{tabularx}
\end{table}
