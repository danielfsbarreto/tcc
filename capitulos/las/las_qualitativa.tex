\chapter{Análise qualitativa de lições aprendidas}

Após a seleção do conjunto de trabalhos primários desta pesquisa (16 artigos científicos e 14 relatos de experiência), as lições aprendidas reportadas por eles foram devidamente mapeadas e categorizadas. Neste capítulo encontra-se o resultado deste mapeamento.

\section{Mapeamento e categorização}
Para obter-se uma maior visibilidade do que foi encontrado, as diversas lições aprendidas reportadas pela base da literatura científica e relatos de experiência foram multidimensionalmente mapeadas.

Foi considerada a existência real de uma categoria quando pelo menos 6 trabalhos (20\% do material analisado) referenciavam alguma lição aprendida que se encaixasse naquela categoria (Tabela \ref{tab:mapeamentoCategorias}).

\begin{table}[H]
	\centering
	\begin{tabularx}{\linewidth}{ | p{6cm} | X | X | }
		\hline 
		\textbf{Categorias} & \textbf{Artigos científicos} & \textbf{Relatos de experiência} \\ 
		\hline 
		Experiência, treinamento e aprendizado & \cite{Hajjdiab2011}, \cite{Block2011}, \cite{Adobe2012}, \cite{Cisco2011}, \cite{Lapham2012}, \cite{Eunha2012}, \cite{Claudia2013}, \cite{Asnawi2012}, \cite{Fitzgerald2013} & \cite{Stefano2013}, \cite{Rodrigues2013}, \cite{Bastos2013}, \cite{Maciel2013}, \cite{Karaj2013}, \cite{Piegas2012}, \cite{Vieira2013} \\ 
		\hline 
		Planejamento e gerenciamento de backlog & \cite{Hajjdiab2011}, \cite{Fitzgerald2013}, \cite{Block2011}, \cite{Adobe2012}, \cite{Bustard2013}, \cite{Korhonen2010}, \cite{Claudia2013} & \cite{Piegas2012}, \cite{Hui2013}, \cite{Parzinello2012} \\ 
		\hline 
		Apoio gerencial e dos clientes & \cite{Hajjdiab2011}, \cite{Cisco2011}, \cite{Claudia2013}, \cite{Arikpo2011} & \cite{Parzinello2012}, \cite{Stefano2013}, \cite{Bastos2013}, \cite{Maciel2013}, \cite{Srinath2012}, \cite{Piegas2012} \\ 
		\hline 
		Customização e adaptabilidade & \cite{Hajjdiab2011}, \cite{Block2011}, \cite{Asnawi2012}, \cite{Fitzgerald2013}, \cite{Bustard2013}, \cite{Microsoft2013}, \cite{Lapham2012}, \cite{Claudia2013}, \cite{Nokia2013} & \cite{Piegas2012}, \cite{Hui2013}, \cite{Rodrigues2013}, \cite{Bastos2013}, \cite{Maciel2013}, \cite{Ahmed2008}, \cite{Sahota2012}, \cite{Vieira2013} \\ 
		\hline 
		Confiança do time & \cite{Block2011}, \cite{Asnawi2012}, \cite{Claudia2013}, \cite{Nokia2013} & \cite{Parzinello2012}, \cite{Ahmed2008}, \cite{Piegas2012}, \cite{Bastos2013} \\ 
		\hline 
		Engajamento, comprometimento, disciplina e trabalho em equipe & \cite{Block2011}, \cite{Asnawi2012}, \cite{Lapham2012}, \cite{Microsoft2013}, \cite{Claudia2013}, \cite{Nokia2013}, \cite{Adobe2012}, \cite{Fitzgerald2013} & \cite{Piegas2012}, \cite{Parzinello2012}, \cite{Stefano2013}, \cite{Rodrigues2013}, \cite{Maciel2013}, \cite{Queiroz2013}, \cite{Bastos2013}, \cite{Ahmed2008} \\ 
		\hline 
		Aspectos técnicos e tecnológicos & \cite{Block2011}, \cite{Microsoft2013}, \cite{Korhonen2010}, \cite{Cisco2011}, \cite{Lapham2012}, \cite{Eunha2012}, \cite{Fitzgerald2013}, \cite{Arikpo2011}, \cite{Bustard2013}, \cite{Radha2012}, \cite{Nokia2013} & \cite{Piegas2012}, \cite{Queiroz2013}, \cite{Stefano2013}, \cite{Karaj2013} \\ 
		\hline 
		Compartilhamento de conhecimento & \cite{Asnawi2012}, \cite{Cisco2011}, \cite{Lapham2012}, \cite{Radha2012}, \cite{Eunha2012}, \cite{Ericsson2013} & \cite{Valerio2013}, \cite{Vieira2013}, \cite{Queiroz2013}, \cite{Bastos2013}, \cite{Maciel2013} \\ 
		\hline 
		Velocidade de entrega e produtividade & \cite{Adobe2012}, \cite{Fitzgerald2013}, \cite{Microsoft2013}, \cite{Cisco2011}, \cite{Korhonen2010}, \cite{Eunha2012}, \cite{Claudia2013} & \cite{Stefano2013}, \cite{Queiroz2013}, \cite{Maciel2013}, \cite{Hui2013}, \cite{Ahmed2008}, \cite{Piegas2012} \\ 
		\hline 
		Qualidade do produto final & \cite{Adobe2012}, \cite{Fitzgerald2013}, \cite{Bustard2013}, \cite{Lapham2012}, \cite{Eunha2012}, \cite{Claudia2013}, \cite{Korhonen2010} & \cite{Parzinello2012}, \cite{Maciel2013}, \cite{Ahmed2008} \\ 
		\hline 
		Tamanho da organização & \cite{Bustard2013}, \cite{Microsoft2013}, \cite{Claudia2013}, \cite{Korhonen2010}, \cite{Ericsson2013} & \cite{Maciel2013} \\ 
		\hline 
		Quebra de paradigma & \cite{Hajjdiab2011}, \cite{Block2011}, \cite{Korhonen2010}, \cite{Lapham2012}, \cite{Arikpo2011} & \cite{Stefano2013}, \cite{Bastos2013}, \cite{Maciel2013}, \cite{Parzinello2012}, \cite{Hui2013}, \cite{Ahmed2008}, \cite{Sahota2012} \\ 
		\hline 
		Comunicação remota & \cite{Adobe2012}, \cite{Microsoft2013}, \cite{Korhonen2010}, \cite{Radha2012} & \cite{Rodrigues2013}, \cite{Vieira2013}, \cite{Bastos2013}, \cite{Maciel2013} \\ 
		\hline 
		Cultura organizacional & \cite{Bustard2013}, \cite{Microsoft2013}, \cite{Claudia2013}, \cite{Nokia2013}, \cite{Eunha2012}, \cite{Fitzgerald2013} & \cite{Rodrigues2013}, \cite{Bastos2013}, \cite{Srinath2012}, \cite{Maciel2013}, \cite{Sahota2012} \\ 
		\hline 
	\end{tabularx}
	\caption{Mapeamento de categorias de lições aprendidas e suas respectivas referências}
	\label{tab:mapeamentoCategorias}
\end{table}

\section{Lições aprendidas}

\subsection{Experiência, treinamento e aprendizado}

\begin{table}[H]
	\centering
	\begin{tabularx}{\linewidth}{ | X | p{5cm} | } \hline \textbf{Lições aprendidas} & \textbf{Referências} \\ \hline
		É muito difícil aventurar-se em ágil sem um Agile Coach & \cite{Hajjdiab2011}, \cite{Block2011}, \cite{Adobe2012}, \cite{Cisco2011}, \cite{Lapham2012}, \cite{Eunha2012}, \cite{Claudia2013}, \cite{Stefano2013}, \cite{Rodrigues2013}, \cite{Bastos2013}, \cite{Maciel2013}, \cite{Karaj2013} \\ \hline
		Vivenciar um projeto-piloto é uma prática muito eficiente para se adquirir experiência & \cite{Hajjdiab2011}, \cite{Cisco2011}, \cite{Rodrigues2013}, \cite{Maciel2013} \\ \hline
		É preciso prática, não apenas estudo & \cite{Hajjdiab2011}, \cite{Asnawi2012}, \cite{Claudia2013}, \cite{Piegas2012}, \cite{Vieira2013} \\ \hline
		Trabalhar com equipes não-agéis pode atrapalhar o andamento do projeto & \cite{Adobe2012}, \cite{Maciel2013} \\ \hline
	\end{tabularx}
\end{table}

\subsection{Planejamento e gerenciamento de backlog}

\begin{table}[H]
	\centering
	\begin{tabularx}{\linewidth}{ | X | p{5cm} | } \hline \textbf{Lições aprendidas} & \textbf{Referências} \\ \hline
		Planejar apenas quando necessário & \cite{Hajjdiab2011}, \cite{Fitzgerald2013}, \cite{Piegas2012}, \cite{Hui2013}, \cite{Parzinello2012} \\ \hline
		O processo de adoção em si precisa ser bem planejado & \cite{Hajjdiab2011} \\ \hline
		Priorizar o backlog é uma tarefa complicada para times inexperientes & \cite{Block2011} \\ \hline
		É difícil lidar com o aumento de escopo & \cite{Block2011} \\ \hline
		É preciso aprender a quebrar o backlog da forma correta & \cite{Adobe2012}, \cite{Hui2013}, \cite{Parzinello2012} \\ \hline
		A coleta e gerência de requisitos ocorre de forma bem diferente do habitual & \cite{Bustard2013}, \cite{Korhonen2010}, \cite{Claudia2013}, \cite{Piegas2012}, \cite{Hui2013} \\ \hline
	\end{tabularx}
\end{table}

\subsection{Apoio gerencial e dos clientes}

\begin{table}[H]
	\centering
	\begin{tabularx}{\linewidth}{ | X | p{5cm} | } \hline \textbf{Lições aprendidas} & \textbf{Referências} \\ \hline
		Template & \cite{} \\ \hline
	\end{tabularx}
\end{table}

\subsection{Customização e adaptabilidade}

\begin{table}[H]
	\centering
	\begin{tabularx}{\linewidth}{ | X | p{5cm} | } \hline \textbf{Lições aprendidas} & \textbf{Referências} \\ \hline
		Template & \cite{} \\ \hline
	\end{tabularx}
\end{table}

\subsection{Confiança do time}

\begin{table}[H]
	\centering
	\begin{tabularx}{\linewidth}{ | X | p{5cm} | } \hline \textbf{Lições aprendidas} & \textbf{Referências} \\ \hline
		Template & \cite{} \\ \hline
	\end{tabularx}
\end{table}

\subsection{Engajamento, comprometimento, disciplina e trabalho em equipe}

\begin{table}[H]
	\centering
	\begin{tabularx}{\linewidth}{ | X | p{5cm} | } \hline \textbf{Lições aprendidas} & \textbf{Referências} \\ \hline
		Template & \cite{} \\ \hline
	\end{tabularx}
\end{table}

\subsection{Aspectos técnicos e tecnológicos}

\begin{table}[H]
	\centering
	\begin{tabularx}{\linewidth}{ | X | p{5cm} | } \hline \textbf{Lições aprendidas} & \textbf{Referências} \\ \hline
		Template & \cite{} \\ \hline
	\end{tabularx}
\end{table}

\subsection{Compartilhamento de conhecimento}

\begin{table}[H]
	\centering
	\begin{tabularx}{\linewidth}{ | X | p{5cm} | } \hline \textbf{Lições aprendidas} & \textbf{Referências} \\ \hline
		Template & \cite{} \\ \hline
	\end{tabularx}
\end{table}

\subsection{Velocidade de entrega e produtividade}

\begin{table}[H]
	\centering
	\begin{tabularx}{\linewidth}{ | X | p{5cm} | } \hline \textbf{Lições aprendidas} & \textbf{Referências} \\ \hline
		Template & \cite{} \\ \hline
	\end{tabularx}
\end{table}

\subsection{Qualidade do produto final}

\begin{table}[H]
	\centering
	\begin{tabularx}{\linewidth}{ | X | p{5cm} | } \hline \textbf{Lições aprendidas} & \textbf{Referências} \\ \hline
		Template & \cite{} \\ \hline
	\end{tabularx}
\end{table}

\subsection{Tamanho da organização}

\begin{table}[H]
	\centering
	\begin{tabularx}{\linewidth}{ | X | p{5cm} | } \hline \textbf{Lições aprendidas} & \textbf{Referências} \\ \hline
		Template & \cite{} \\ \hline
	\end{tabularx}
\end{table}

\subsection{Quebra de paradigma}

\begin{table}[H]
	\centering
	\begin{tabularx}{\linewidth}{ | X | p{5cm} | } \hline \textbf{Lições aprendidas} & \textbf{Referências} \\ \hline
		Template & \cite{} \\ \hline
	\end{tabularx}
\end{table}

\subsection{Comunicação remota}

\begin{table}[H]
	\centering
	\begin{tabularx}{\linewidth}{ | X | p{5cm} | } \hline \textbf{Lições aprendidas} & \textbf{Referências} \\ \hline
		Template & \cite{} \\ \hline
	\end{tabularx}
\end{table}

\subsection{Cultura organizacional}

\begin{table}[H]
	\centering
	\begin{tabularx}{\linewidth}{ | X | p{5cm} | } \hline \textbf{Lições aprendidas} & \textbf{Referências} \\ \hline
		Template & \cite{} \\ \hline
	\end{tabularx}
\end{table}
