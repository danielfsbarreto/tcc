\chapter{Validação, conclusões e trabalhos futuros}

Para assegurar-se da relevância das categorias de lições aprendidas mapeadas no capítulo anterior, elas foram validadas por profissionais com experiência em adoção ágil atuantes do mercado de TI brasileiro. Além disso, neste capítulo encontra-se o fechamento da pesquisa, com suas conclusões e possíveis trabalhos futuros.

\section{Método de validação}

Após uma revisão de literatura baseada no método de revisão sistemática sobre artigos publicados em revistas e conferências científicas, complementada por  uma pesquisa exploratória de relatos de experiência nas principais conferências nacionais e internacionais no assunto, um conjunto de lições aprendidas foram coletadas. As mesmas foram agrupadas considerando suas similaridades, em 14 categorias/aspectos específicos.

Com o intuito de validar a relevância dessas lições aprendidas propostas por esta pesquisa, um questionário foi elaborado para coletar a opinião de profissionais que vivenciaram a adoção organizacional da abordagem ágil. São 15 questões de múltipla escolha cujo foco é a criticidade de cada categoria/aspecto para que a adoção de Ágil em empresas de desenvolvimento de software ocorra com sucesso.

\subsection{Perfil dos profissionais consultados}

Foram recebidas [X] respostas de profissionais que têm em média [Y] anos de experiência com desenvolvimento de software e cerca de [Z] anos com Ágil. Para ter um melhor controle com relação à competência e ao nível de experiência dos profissionais consultados, o questionário não foi disponibilizado publicamente.

\subsection{Questionário: perguntas e respostas}

\subsubsection{Qual nível de prioridade você daria para a experiência, treinamento e aprendizado dos envolvidos?}

[Graph goes here]

\subsubsection{Qual nível de prioridade você daria para o planejamento e gerenciamento do backlog do projeto?}

[Graph goes here]

\subsubsection{Qual nível de prioridade você daria para o apoio gerencial e dos clientes?}

[Graph goes here]

\subsubsection{Qual nível de prioridade você daria para a capacidade de customização e adaptabilidade do projeto?}

[Graph goes here]

\subsubsection{Qual nível de prioridade você daria para o nível de confiança do time?}

[Graph goes here]

\subsubsection{Qual nível de prioridade você daria para o nível de engajamento, comprometimento, disciplina e trabalho em equipe dos integrantes do projeto?}

[Graph goes here]

\subsubsection{Qual nível de prioridade você daria para os aspectos técnicos e tecnológicos utilizados?}

[Graph goes here]

\subsubsection{Qual nível de prioridade você daria para o hábito do compartilhamento de conhecimento dentro e fora do projeto?}

[Graph goes here]

\subsubsection{Qual nível de prioridade você daria para a velocidade de entrega e a produtividade da equipe?}

[Graph goes here]

\subsubsection{Qual nível de prioridade você daria para a busca pela melhoria na qualidade do produto final?}

[Graph goes here]

\subsubsection{Qual nível de prioridade você daria para os esforços com a quebra de paradigma necessária para a implementação de Ágil?}

[Graph goes here]

\subsubsection{Qual nível de prioridade você daria para o esforço para se lidar com comunicação remota?}

[Graph goes here]

\subsubsection{Qual nível de prioridade você daria para o alinhamento dos Princípios Ágeis com a cultura da organização?}

[Graph goes here]

\subsubsection{Se você tivesse que selecionar apenas 5 das categorias relacionadas anteriormente, quais você ressaltaria como as mais críticas para serem consideradas por uma organização na adoção da abordagem ágil?}

[Graph goes here]

\section{Conclusões}


\section{Trabalhos futuros}
