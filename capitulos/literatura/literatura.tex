
\chapter{Revisão da Literatura}
	Serão apresentados neste capítulo os seguintes pontos: a estratégia de pesquisa adotada neste projeto de conclusão de curso, os desafios e as lições aprendidas documentadas por organizações de desenvolvimento de software que tentaram (com sucesso ou não) adotar ágil.
	\section{Método de pesquisa}
		O processo de pesquisa utilizado neste projeto foi baseado no método chamado revisão sistemática. Não foi necessário ser completamente fidedigno ao método dado a restrições de tempo, contingente (este foi um trabalho desenvolvido individualmente) e nível de detalhe.
		
		Segundo Barbara Kitchenham \cite{Barbara04}, uma revisão sistemática da literatura consiste em identificar, avaliar e interpretar todas pesquisas disponíveis relevantes a uma determinada questão, tópico ou área de interesse. Este tipo de pesquisa requer a definição de alguns pontos:
		\begin{itemize}
			\item A(s) questão(ões) a ser(em) respondida(s) ao final da pesquisa
			\item A estratégia utilizada (fontes consultadas, termos buscados, etc.)
			\item Critérios de seleção e exclusão de artigos
		\end{itemize}
		A necessidade da adoção de uma revisão sistemática decorre da exigência de se sumarizar todo um conjunto de informações existentes no que tange à adoção de metodologias ágeis por parte de empresas de desenvolvimento de software. A partir deste estudo, pode-se criar um arcabouço para a elaboração dos questionários que foram utilizados neste projeto.
		\subsection{Definição da pergunta}
			O primeiro passo foi definir a questão a ser respondida ao final da pesquisa: ``\textit{Quais os desafios e benefícios da adoção de metodologias ágeis por parte de empresas de desenvolvimento de software?}"
		\subsection{Bases de dados relevantes}
			Após a definição da pergunta a ser respondida, o próximo passo foi definir as fontes de informação mais confiáveis e relevantes para o tema (Tabela \ref{tab:basesDeDados}).
			\begin{table}[H]
				\centering
				\begin{tabular}{| l | r |} \hline \textbf{Nome} & \textbf{Referência} \\ \hline
					CAPES & http://www.capes.gov.br/ \\ \hline
					ACM & http://www.acm.org/ \\ \hline
					IEEE & http://ieeexplore.ieee.org/ \\ \hline
					Google Scholar & http://scholar.google.com/ \\ \hline
				\end{tabular}
				\caption{Resumo das bases de dados utilizadas na pesquisa}
				\label{tab:basesDeDados}
			\end{table}
			\nomenclature{CAPES}{Coordenação de Aperfeiçoamento de Pessoal de Nível Superior}%
			\nomenclature{ACM}{Association for Computing Machinery}%
			\nomenclature{IEEE}{Instituto de Engenheiros Eletricistas e Eletrônicos}%
		\subsection{Palavras-chave}
			Neste passo, foram definidas as palavras-chave (Tabela \ref{tab:palavrasChave}) buscadas nas bases de dados selecionadas.
			\begin{table}[H]
				\centering
				\begin{tabular}{| l |} \hline \textbf{Palavras-chave} \\ \hline
					Agile Adoption \\ \hline
					Agile Enterprise \\ \hline
				\end{tabular}
				\caption{Conjunto de palavras-chave utilizadas na pesquisa}
				\label{tab:palavrasChave}
			\end{table}
		\subsection{Critérios de exclusão}
			//Área para definir critérios de descarte após a pesquisa ter sido feita (colocar números)
			\begin{table}[H]
				\centering
				\begin{tabular}{| c | l | r |} \hline \textbf{Critério de exclusão} & \textbf{Base de dados}  & \textbf{Quantidade} \\ \hline
					\multirow{4}{*}{-}
						& ACM & 1 \\ \cline{2-3}
						& CAPES & 1 \\ \cline{2-3}
						& Google Scholar & 1 \\ \cline{2-3}
						& IEEE & 1 \\ \cline{2-3}
					\hline \hline
					\multirow{4}{*}{Artigos entre 2010 e 2013} 
						& ACM & 1 \\ \cline{2-3}
						& CAPES & 1 \\ \cline{2-3}
						& Google Scholar & 1 \\ \cline{2-3}
						& IEEE & 1 \\ \cline{2-3}
					\hline \hline
					\multirow{4}{*}{Palavras-chave no título e/ou resumo} 
						& ACM & 1 \\ \cline{2-3}
						& CAPES & 1 \\ \cline{2-3}
						& Google Scholar & 1 \\ \cline{2-3}
						& IEEE & 1 \\ \cline{2-3}
					\hline \hline
					\multirow{4}{*}{Análise crítica}
						& ACM & 1 \\ \cline{2-3}
						& CAPES & 1 \\ \cline{2-3}
						& Google Scholar & 1 \\ \cline{2-3}
						& IEEE & 1 \\ \cline{2-3}
					\hline
				\end{tabular}
				\caption{Quantidade de materiais encontrados em cada passo da pesquisa}
				\label{tab:quantidadeDeMateriais}
			\end{table}

