\chapter{Revisão da Literatura}
	Serão apresentados neste capítulo os seguintes pontos: a estratégia de pesquisa adotada neste projeto de conclusão de curso, os desafios e as lições aprendidas documentadas por organizações de desenvolvimento de software que tentaram (com sucesso ou não) adotar ágil.
	\section{Método de pesquisa}
		O processo de pesquisa utilizado neste projeto foi baseado no método chamado revisão sistemática. Não foi necessário ser completamente fidedigno ao método dado a restrições de tempo, contingente (este foi um trabalho desenvolvido individualmente) e nível de detalhe.
		
		Segundo Barbara Kitchenham \cite{Barbara04}, uma revisão sistemática da literatura consiste em identificar, avaliar e interpretar todas pesquisas disponíveis relevantes a uma determinada questão, tópico ou área de interesse. Este tipo de pesquisa requer a definição de alguns pontos:
		\begin{itemize}
			\item A(s) questão(ões) a ser(em) respondida(s) ao final da pesquisa
			\item A estratégia utilizada (fontes consultadas, termos buscados, etc.)
			\item Critérios de seleção e exclusão de material
		\end{itemize}
		A necessidade da adoção de uma revisão sistemática decorre da exigência de se sumarizar todo um conjunto de informações existentes no que tange à adoção de metodologias ágeis por parte de empresas de desenvolvimento de software. A partir deste estudo, pode-se criar um arcabouço para a elaboração dos questionários que foram utilizados neste projeto.
		\subsection{Definição da pergunta}
			O primeiro passo foi definir a questão a ser respondida ao final da pesquisa: ``\textit{Quais as principais lições aprendidas com a adoção de metodologias ágeis por parte de empresas de desenvolvimento de software na atualidade?}"
		\subsection{Bases de dados relevantes}
			Após a definição da pergunta a ser respondida, o próximo passo foi definir as fontes de informação mais confiáveis e relevantes para o tema (Tabela \ref{tab:basesDeDados}).
			\begin{table}[H]
				\centering
				\begin{tabular}{| l | r |} \hline \textbf{Nome} & \textbf{Referência} \\ \hline
					CAPES & http://www.capes.gov.br/ \\ \hline
					ACM & http://www.acm.org/ \\ \hline
					IEEE & http://ieeexplore.ieee.org/ \\ \hline
					Google Scholar & http://scholar.google.com/ \\ \hline
					Springer Link & http://link.springer.com/ \\ \hline
				\end{tabular}
				\caption{Bases de dados consultadas na pesquisa}
				\label{tab:basesDeDados}
			\end{table}
			\nomenclature{CAPES}{Coordenação de Aperfeiçoamento de Pessoal de Nível Superior}%
			\nomenclature{ACM}{Association for Computing Machinery}%
			\nomenclature{IEEE}{Instituto de Engenheiros Eletricistas e Eletrônicos}%
		\subsection{Palavras-chave}
			Neste passo, foram definidas as palavras-chave (Tabela \ref{tab:palavrasChave}) buscadas nas bases de dados selecionadas.
			\begin{table}[H]
				\centering
				\begin{tabular}{| l |} \hline \textbf{Palavras-chave} \\ \hline
					Agile Adoption \\ \hline
				\end{tabular}
				\caption{Conjunto de palavras-chave utilizadas na pesquisa}
				\label{tab:palavrasChave}
			\end{table}
		\subsection{Critérios de exclusão}
			A base da literatura foi selecionada através do critério de exclusão de artigos. Como existem diversos materiais publicados na área de adoção ágil nas bases de dados selecionadas, foram encontrados cerca de 1000 artigos utilizando-se o conjunto de palavras-chave \textit{``Agile Adoption"}.

			Porém, muitos dos dados coletados nesta pesquisa preliminar eram muito antigos, poderiam não fazer mais muito sentido nos dias atuais. Sendo assim, foi adicionado um segundo critério de exclusão: a data de publicação. Apenas materiais atuais (entre 2010 e 2013) seriam analisados. O contingente de trabalhos passou para, em média, 550 artigos.

			Outro ponto importante a ser levado em consideração é a relevância do tema nos trabalhos restantes. O próximo filtro aplicado foi o de se encontrar as palavras-chave no título ou resumo. Com isso, o número de trabalhos reduziu bastante, para cerca de 50 artigos. Um ponto que vale a pena ser mencionado é que, nesta etapa do processo, caso o critério fosse levado extremamente à risca, bons materiais seriam eliminados. Este fator foi analisado na etapa posterior.

			O último critério de exclusão utilizado nesta pesquisa foi uma análise qualitativa do material restante. Após uma revisão, artigo por artigo, 16 foram escolhidos, estando 2 deles fora do critério de exclusão anterior. A Tabela  \ref{tab:quantidadeDeMateriais} mostra com detalhes a quantidade de materiais selecionados em cada etapa.
			\begin{table}[H]
				\centering
				\begin{tabular}{| c | l | r |} \hline \textbf{Critério de exclusão} & \textbf{Base de dados}  & \textbf{Quantidade} \\ \hline
					\multirow{4}{*}{-}
						& ACM & 106 \\ \cline{2-3}
						& CAPES & 51 \\ \cline{2-3}
						& Google Scholar & 849 \\ \cline{2-3}
						& IEEE & 30 \\ \cline{2-3}
						& Springer Link & 6 \\ \cline{2-3}
					\hline \hline
					\multirow{4}{*}{Artigos entre 2010 e 2013} 
						& ACM & 26 \\ \cline{2-3}
						& CAPES & 37 \\ \cline{2-3}
						& Google Scholar & 459 \\ \cline{2-3}
						& IEEE & 16 \\ \cline{2-3}
						& Springer Link & 5 \\ \cline{2-3}
					\hline \hline
					\multirow{4}{*}{Palavras-chave no título e/ou resumo} 
						& ACM & 1 \\ \cline{2-3}
						& CAPES & 2 \\ \cline{2-3}
						& Google Scholar & 38 \\ \cline{2-3}
						& IEEE & 14 \\ \cline{2-3}
						& Springer Link & 0* \\ \cline{2-3}
					\hline \hline
					\multirow{4}{*}{Análise crítica}
						& ACM & 0 \\ \cline{2-3}
						& CAPES & 0 \\ \cline{2-3}
						& Google Scholar & 4 \\ \cline{2-3}
						& IEEE & 10 \\ \cline{2-3}
						& Springer Link & 2 \\ \cline{2-3}
					\hline
				\end{tabular}
				\caption{Quantidade de material encontrado em cada passo da pesquisa}
				\label{tab:quantidadeDeMateriais}
			\end{table}

	\section{Materiais selecionados}