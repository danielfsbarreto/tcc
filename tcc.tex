%=======================================================================
% Template para projeto de pesquisa
% Programa de Pós Graduação em Informática Aplicada da Universidade Federal 
% Rural de Pernambuco
%=======================================================================
\documentclass[a4paper,11pt]{article}
\usepackage[left=3.0cm,right=2.5cm,top=3.0cm]{geometry}
%espaçamento entre parágrafos
\setlength{\parindent}{.0in}
\parskip 10pt        
\usepackage[utf8]{inputenc}
\usepackage[brazilian]{babel}
\usepackage{hyperref}
\usepackage{color}
\usepackage{graphicx}
\usepackage{float}
        
\newcommand{\D}{\displaystyle}
\newcommand{\mc}{\multicolumn}
\newcommand{\ft}{\footnotesize}
\newcommand{\Sc}{\scriptsize}
\newcommand{\rset}{\mathbb{R}}
\newcommand{\zset}{\mathbb{Z}}

\newcommand\SB[1]{{\color{cyan} #1}}
\newcommand\JC[1]{{\color{red} #1}}
\newcommand\AO[1]{{\color{green} #1}}


\begin{document}
\pagestyle {empty}


%CAPA
\vspace*{-2cm}
\begin{figure}[h]
\leavevmode
\begin{minipage}[t]{\textwidth}
\includegraphics[scale=0.7]{images/logo-ufrpe.eps}
\end{minipage}
\end{figure}
\vspace*{-3.0cm}
{\bf
\begin{center}
{
\hspace*{0cm} 	MINISTÉRIO DA EDUCAÇÃO E DO DESPORTO \\
\hspace*{.2in} UNIVERSIDADE FEDERAL RURAL DE PERNAMBUCO \\
\hspace*{.2in} BACHARELADO EM SISTEMAS DE INFORMAÇÃO} \\
\end{center}}
\vspace{0.0cm}
\noindent
\begin{figure}[h]
\centering
\includegraphics[scale=0.5]{images/Logo-bsi-presencial-v3-amp.eps}
\end{figure}
\vspace*{2.0cm}
\begin{center}
{\Large \bf  Projeto de Conclusão de Curso}\\[1cm]
{\Large \bf Título: Desafios e lições aprendidas da transição para o desenvolvimento ágil de software em empresas brasileiras} \\[3cm]
\end{center}
{\Large  Estudante: Daniel Filype Silva Barreto}\\[6mm]
{\Large  Orientador: Teresa Maciel}\\[6mm]
\vspace{3.0cm}
\begin{center}
{\large {\bf Recife}\\[6mm]
Março de 2014}
\end{center}
\newpage
\pagestyle {plain}
\setcounter{page}{0} \pagenumbering{arabic}

%REVISÃO SISTEMÁTICA
\section{Revisão Sistemática}
	Serão apresentados neste capítulo os seguintes pontos: a estratégia de pesquisa adotada neste projeto de conclusão de curso, os desafios e as lições aprendidas documentadas por organizações de desenvolvimento de software que tentaram (com sucesso ou não) adotar ágil.
	\subsection{Método de Pesquisa}
		O processo de pesquisa deste projeto foi realizado através de  uma revisão sistemática. Segundo Barbara Kitchenham \cite{Barbara04}, uma revisão sistemática da literatura consiste em identificar, avaliar e interpretar todas pesquisas disponíveis relevantes a uma determinada questão, tópico ou área de interesse. Este tipo de pesquisa requer a definição de alguns pontos:
		\begin{itemize}
			\item A(s) questão(ões) a ser(em) respondida(s) ao final da pesquisa
			\item A estratégia utilizada (fontes consultadas, termos buscados, etc.)
			\item Critérios de seleção e exclusão de artigos
			\item Procedimentos de avaliação de qualidade de material
		\end{itemize}
		A necessidade da adoção de uma revisão sistemática decorre da exigência de se sumarizar todo um conjunto de informações existentes no que tange à adoção de metodologias ágeis por parte de empresas de desenvolvimento de software. A partir deste estudo, pode-se criar um arcabouço para a elaboração dos questionários que foram utilizados neste projeto.
		\subsubsection{Definição da pergunta}
			O primeiro passo foi definir a questão a ser respondida ao final da pesquisa: ``\textit{Quais os desafios e benefícios da adoção de metodologias ágeis por parte de empresas de desenvolvimento de software?}"
		\subsubsection{Bases de dados relevantes}
			Após a definição da pergunta a ser respondida, o próximo passo foi definir as fontes de informação mais confiáveis e relevantes para o tema (Tabela \ref{tab:basesDeDados}).
			\begin{table}[H]
				\centering
				\begin{tabular}{ | l | c | c | } \hline \textbf{Nome} & \textbf{Tipo} & \textbf{Referência} \\ \hline
					CAPES & Online & http://www.capes.gov.br/ \\ \hline
					ACM & Online & http://www.acm.org/ \\ \hline
					IEEE & Online & http://ieeexplore.ieee.org/ \\ \hline
					Google Scholar & Online & http://scholar.google.com/ \\ \hline
					Agile Brazil & Conferência & - \\ \hline
					SBQS & Conferência & - \\ \hline
					ICSE & Conferência & - \\ \hline
					WBMA & Conferência & - \\ \hline
					Agile Conference & Conferência & - \\ \hline
					XP Conference & Conferência & - \\ \hline
				\end{tabular}
				\caption{Resumo das bases de dados utilizadas na pesquisa}
				\label{tab:basesDeDados}
			\end{table}
		\subsubsection{Palavras-chave}
			Neste passo, foram definidas as palavras-chave (Tabela \ref{tab:palavrasChave}) buscadas nas bases de dados selecionadas.
			\begin{table}[H]
				\centering
				\begin{tabular}{ | l | } \hline \textbf{Palavras-chave} \\ \hline
					Agile Adoption \\ \hline
					Agile Enterprise \\ \hline
					Survey \\ \hline
					Challenges \\ \hline
					Lessons learned \\ \hline
				\end{tabular}
				\caption{Conjunto de palavras-chave utilizadas na pesquisa}
				\label{tab:palavrasChave}
			\end{table}
		\subsubsection{Seleção da base da literatura}
			//Área para definir critérios de descarte após a pesquisa ter sido feita (colocar números)
	\subsection{Desafios}
		//Área para expor desafios encontrados
	\subsection{Lições aprendidas}
		//Área para expor lições aprendidas encontradas

%BIBLIOGRAFIA
\newpage
\bibliographystyle{plain}
\bibliography{referencias}


\end{document}
